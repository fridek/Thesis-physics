\chapter{Summary}
\label{cha:summary}

Conducted experiments show a gap between JavaScript and C++ performance. Significant language design differences result in code that is often easier to write but also easier to abuse. Benchmarks show differences between 15\% and 100\% overhead for correctly designed JavaScript code and over 500\% for incorrect patterns.  Considering Moore's law stating that computers double speed every 18 months it safe to say that JavaScript is very close to being suitable for any type of development. 

Conducted experiments show clear pattern regarding dynamic variable types in JavaScript. Whenever boxing and unboxing happens, JIT compilation is not able to properly optimise code and bring it up to performance of C++. This affects both simple variables and properties end is especially visible for numbers. Transitions between integer and floats are expensive while easy to overlook.

Types affect significantly also method calls cost. Keeping methods monomorphic in core parts of physics engine is very important. Additional cost of polymorphism of parameters is not only boxing and unboxing of parameters but also time spent of optimising and deoptimising compiled method which makes initial warmup of engine longer. Exporting well defined methods to be called from polymorphic ones is an easy workaround for this performance bottleneck.

Lastly, memory management proven to be one of the most important problems in JavaScript. Automated garbage collection connected with popular pattern of creating and returning of arrays is an important problem. Memory allocation of objects is also a bottleneck, but not very dissimilar to one in C++. As shown in second version of particle system, usage of object pools and changing architecture to avoid array creation are techniques that can be employed to fight with it. It is worth mentioning that while garbage collection always introduces some overhead it is reasonable to avoid it at all costs. Sphere collision system with octree partitioning is also introducing and destroying objects, but overhead is significantly smaller than gained speedup. Advice for memory operations is to avoid objects living only for a single frame i.e. temporary variables and helpers. Long living objects are in general unavoidable and should be used whenever suitable.

General advice for programming in JavaScript is to use techniques similar to those found in asm.js - keep types static, method calls monomorphic and work carefully with memory.

Conducted tests show that while gap between JavaScript and native application exists and is not insignificant, there is a lot of potential in such approach. It is expected, that with growing community and interest from game industry new games will be released on browser within few years. Performance issues may prevent works on AAA titles, but companies focused more on social aspect of games and new trends in monetisation may create games targeted for different users. With capabilities of browsers equal to having 18 months older machine, less graphically demanding titles like The Sims or World or Warcraft may certainly be ported to run in JavaScript.

