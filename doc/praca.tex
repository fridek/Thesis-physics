
% \documentclass[pdflatex,11pt]{aghdpl}
\documentclass[en]{ujdpl}               % przy kompilacji programem latex
% \documentclass[pdflatex,en]{aghdpl}  % praca w jezyku angielskim

\usepackage[polish]{babel}
\usepackage[cp1250]{inputenc}
\usepackage{ae}

\usepackage[T1]{fontenc}
\renewcommand*\familydefault{\sfdefault} %% Only if the base font of the document is to be sans serif

\usepackage[justification=centering]{caption}

\usepackage{hyperref} 
\usepackage{url}
\usepackage{latexsym}
\usepackage{textcomp}

\usepackage{titlesec}
\usepackage{graphics} % wlaczanie grafik
\usepackage{wrapfig}

\usepackage{color} % kolory
\usepackage{floatflt}
\usepackage{bookmark}
\frenchspacing
\usepackage{indentfirst}

% dodatkowe pakiety
\usepackage{amsmath}
\usepackage{multirow}
\usepackage{enumerate}
\usepackage{listings}
\usepackage[usenames,dvipsnames]{xcolor}
\lstloadlanguages{xml}

\hyphenpenalty=100000

\definecolor{listinggray}{gray}{0.9}
\definecolor{lbcolor}{rgb}{0.9,0.9,0.9}
\lstset{
	backgroundcolor=\color{lbcolor},
	tabsize=4,
	rulecolor=,
	language=xml,
        basicstyle=\scriptsize,
        upquote=true,
        aboveskip={1.5\baselineskip},
        columns=fixed,
        showstringspaces=false,
        extendedchars=true,
        breaklines=true,
        prebreak = \raisebox{0ex}[0ex][0ex]{\ensuremath{\hookleftarrow}},
        frame=single,
        showtabs=false,
        showspaces=false,
        showstringspaces=false,
        identifierstyle=\ttfamily,
        keywordstyle=\color[rgb]{0,0,1},
        commentstyle=\color[rgb]{0.133,0.545,0.133},
        stringstyle=\color[rgb]{0.627,0.126,0.941},
}


\lstdefinelanguage{JavaScript}{
  keywords={typeof, new, true, false, catch, function, return, null, catch, switch, var, if, in, while, do, else, case, break, this},
  keywordstyle=\color{blue}\bfseries,
  ndkeywords={class, export, boolean, throw, implements, import},
  ndkeywordstyle=\color{}\bfseries,
  identifierstyle=\color{black},
  sensitive=false,
  comment=[l]{//},
  morecomment=[s]{/*}{*/},
  commentstyle=\color{purple}\ttfamily,
  stringstyle=\color{red}\ttfamily,
  morestring=[b]',
  morestring=[b]"
}

\lstset{
   language=JavaScript,
   backgroundcolor=\color{listinggray},
   extendedchars=true,
   basicstyle=\footnotesize\ttfamily,
   showstringspaces=false,
   showspaces=false,
   numbers=left,
   numberstyle=\footnotesize,
   numbersep=9pt,
   tabsize=2,
   breaklines=true,
   showtabs=false,
   captionpos=b
}


% paragraph gap:
\setlength{\parskip}{0.5\baselineskip plus 0.5ex minus 0.2ex}

%---------------------------------------------------------------------------

\author{Sebastian Por�ba}
\shortauthor{Sebastian Por�ba}

\titlePL{Por�wnanie silnik�w fizyki 3D}
\titleEN{Comparison of 3D physics engines}

\shorttitlePL{Por�wnanie silnik�w fizyki 3D} % skr�cona 
\shorttitleEN{Comparison of 3D physics engines}

\thesistypePL{Praca magisterska}
\thesistypeEN{Master of Science Thesis}

\supervisorPL{dr hab. Pawe� W�grzyn}
\supervisorEN{Pawe� W�grzyn Ph.D}

\departmentPL{Instytut Informatyki Stosowanej}
\departmentEN{Department of Applied Computer Science} 

\date{2013}

\facultyPL{Wydzia� Fizyki, Astronomii i Informatyki Stosowanej}
\facultyEN{Faculty of Phycics, Astronomy and Applied Computer Science}

\acknowledgements{Serdecznie podzi�kowania dla ...}

%---------------------------------------------------------------------------

\begin{document}

\titlepages

\setcounter{tocdepth}{1}
\tableofcontents

\clearpage

\newcommand{\sectionbreak}{\clearpage}

\chapter{Introduction}
\label{cha:introduction}

The main objective of presented project is the implementation of parts of 3D engine in a browser environment. Parts of engine are analysed side-by-side with parallel engine compiled from C++. The objective of analysis is comparison of performance and description of possible issues related to the limited browser resources and dynamic features of JavaScript.

Browser-based engine is implemented in JavaScript and analyzed in V8 engine. V8 is maintained by Google and is used in Google Chrome browser. Executable examples are compiled using gcc compiler and are runned on the same platform. For additional comparison EmScripten project is used to automatically generate JavaScript to measure if automated conversion may be as effective as writing code by hand.

Project is based on conference sessions and announcements authored by V8 programmers regarding performance of JavaScript applications. Analysis of available materials is a topic of Chapter 2, where internals of modern engines for dynamic languages are briefly explained. Results of existing works are reproduced and measured to build a base for extension.

Chapter 3 covers particles system found often in graphic engines. It shows techniques related to memory allocation and garbage collection that help to improve performance.

Chapter 4 is focused on sphere collision detection and reaction, with both naive solution and space partitioning using Octree.

Chapter 5 describes Emscripten, project aimed to convert complete C++ projects to JavaScript. Generated code is compared to the one created in previous chapters.



\chapter{Overview of JavaScript and V8 engine architecture}
\label{cha:overview}

Historically, JavaScript was considered untyped language, meaning that values had no types attached to variables, either by programmer or compiler. All variables were of single, unified type and procedures called unboxing and boxing, performed before and after each operation on variable, ensured that it was properly used on machine code level. Complete code source was sent from server to the browser and was parsed and executed on fly. Without types attached to variables, all functions were polymorphic and unstable, since parameters may have carried any type of variable. To solve this problem source code of function was parsed each time it was called, each time generating machine code based on current parameters and variables in scope. This approach, called interpretation, is still present in JavaScript engines and used whenever variables don't match set criteria of stability described later in this chapter.

This paper uses as an engine of choice V8 Crankshaft. Choice was made because it's the only engine available at the moment which provides direct command line access, enabling precise performance measurements of code parsing and execution, without browser context and overhead. Executable file of V8 (named d8) is compiled with consideration of target platform.

\section{JIT compilation -- tracking variable types}
\label{sec:JIT}

As mentioned before, initially JavaScript was treated as untyped language. With release of SpiderMonkey in Firefox 3.5 in 2009 situation has changed. First Just-In-Time compiler for JavaScript, TraceMonkey, was created. Based on works of Prof. Dr. Michael Franz on TraceTrees \footnote{http://www.michaelfranz.com/} JIT compiler was collecting all paths that interpreter took with specific types of variables. A path could split to different methods or if statements. Whenever part of code was executed often enough, path was marked as hot and compiler optimised it for given types. If single path was traversed with different set of types compiler could generate another version of optimised code. When path tuned out to be highly polymorphic optimised versions were removed and interpreter was used as a fallback.
Initial reports shown speedups between 20x to 40x \footnote{http://arstechnica.com/information-technology/2008/08/firefox-to-get-massive-javascript-performance-boost/}.
However, trace JIT turned out to be very complicated to maintain \footnote{https://hacks.mozilla.org/2010/03/improving-javascript-performance-with-jagermonkey/} and eventually was removed from Firefox in 2011.\footnote{http://blog.mozilla.org/nnethercote/2011/11/23/memshrink-progress-report-week-23/}. At the time SpiderMonkey was already equipped with JägerMonkey, JIT engine based on method calls. Instead of collecting complete traces, only method calls are counted. This gives easy track of function parameters and variables in scope.

\begin{figure}[h!]
  \caption{JIT compiler tracking method calls}
  \label{img:jit-1}
  \centering
	\includegraphics[width=8cm]{jit-1}
\end{figure}


\begin{figure}[h!]
  \caption{JIT compiler marking one of methods as hot and recompiling}
  \label{img:jit-2}
  \centering
	\includegraphics[width=8cm]{jit-2}
\end{figure}

This proved to be more effective and simpler approach and now used in all JavaScript engines. In V8 Crankshaft step forward was taken and simple methods are compiled even before any statistics on data types are collected. For compiled methods source code is not stored. Instead procedure called deoptimisation is implemented. Whenever engine detects that compiled code doesn't match actual types of variables, code is deoptimised and either optimised again to match new, better set of variables, or kept in interpreter friendly form.

To track these changes two debug options for V8 are available: --trace-opt and --trace-deopt.

\lstinputlisting[caption=Output from V8 debug run showing optimisation and deoptimisation,label=listing:optdeopt]{optdeopt.txt}


\section{Type interference}
\label{sec:typeinterference}

V8's method of optimising code before it's run relies on type inference. Based on context of variable it's type is guessed. Generated assembler has to support cache miss -- whenever inferred type turns out to be incorrect, new type is assigned and another JIT compilation runs. Types of variables are organised in a tree, where Number object may store both Float or Integer, Integer may store SMI (small int), etc.

\lstinputlisting[caption=Tree of types in JavaScript,label=listing:typestree]{types.txt}

In V8 type inference is tightly connected with JIT compilation and may be tracked with the same flags: --trace-opt and --trace-deopt. 

\section{Hidden classes}
\label{sec:hiddenclasses}

TODO: add paragraph on dictionary mode in objects.

JavaScript is classless language. Object may have defined a prototype which behaves similar to base class in other langauges. However, a property may be added to an Object or it's prototype at any point in runtime. To optimise such dynamic representation engines use a concept of hidden class. Whenever an Object is created it's hidden class is pointed to base, empty Object representation. Then each definition of new property makes a transition on hidden class graph, introducing hidden classes that are not yet defined, as in following example:

\begin{figure}[h!]
  \caption{Initial shape of hidden class for Point}
  \label{img:point0}
  \centering
	\includegraphics[width=8cm]{point0}
\end{figure}
\begin{figure}[h!]
  \caption{Shape of hidden class for Point after x property added}
  \label{img:point1}
  \centering
	\includegraphics[width=8cm]{point1}
\end{figure}
\begin{figure}[h!]
  \caption{Shape of hidden class for Point after y property added}
  \label{img:point2}
  \centering
	\includegraphics[width=8cm]{point2}
\end{figure}

Based on hidden class further JIT compiler optimises methods, to generate even simpler assembly code similar to one compiled from C++. Class shape defines address offsets of Object properties. Thus, hidden class graph is actually a tree, where one class can't be reached in more than one way.

\begin{figure}[h!]
  \caption{Two point representations based on order of declared properties}
  \label{img:point_tree2}
  \centering
	\includegraphics[width=8cm]{point_tree2}
\end{figure} 

At the moment of writing type of property is not tracked in hidden classes. The only exception are primitive values (see Listing \ref{listing:typestree}). In other words, storing an object in property results in the same hidden class regardless of hidden class of this object.

Transitions between hidden classes can be tracked in V8 using flags --trace-generalization tracking when variables are casted to more generic type (e.g. SMI to Integer, or Integer to Number) and --trace-migration (tracking when hidden classes are migrated).

\lstinputlisting[caption=Log of migration trace in V8,label=listing:migration]{migration.txt}

\section{Garbage collection}
\label{sec:garbagecollection}

Memory in JavaScript is managed automatically. Each allocation puts an object on memory heap. First generation of garbage collection traversed the whole tree and freed memory for all inaccessible objects. This type of deallocation is called mark-sweep and is causing taking a long time. Since JavaScript is single-threaded, this operation is blocking all other operations. To improve performance, especially in games, incremental scavange method of garbage collection was introduced. Engine tracks age of objects, allowing to quickly detect objects allocated temporarily (e.g. for a single frame rendered in game). When object is inaccessible, it's queued for deallocation, in chunks that don't cause long UI freezes. \footnote{http://en.wikipedia.org/wiki/Cheney's\_algorithm} \footnote{http://en.wikipedia.org/wiki/Garbage\_collection\_(computer\_science)}

TODO: check and extent

\lstinputlisting[caption=Log of garbage collection in V8,label=listing:gc]{gc.txt}

\chapter{Particle system}
\label{cha:particlesystem}

Particle system is one of most commonly used techniques to simulate smoke, fire, rain and other groups of discrete objects, usually independent from each other. System consist of defined number of emitters, producing lightweight particle objects with certain parameters. Each emitter has defined production ratio and each particle a certain lifespan, resulting in upper limit of total particles on screen. Some systems use also attraction points which enable better control over particles, using equations often similar to those of electrostatic forces.
Such simulation is independent from rendering. The same particle system may be used for different effects with proper configuration.

\begin{figure}[h!]
  \caption{Example rendering of tested particle system}
  \label{img:particles}
  \centering
	\includegraphics[width=16cm]{particles}
\end{figure}

\section{System parameters}
\label{sec:emittersparameters}

Tested system works on two-dimensional Cartesian plane. For purpose of  performance analysis movements are calculated based only on frames rather than actual flow of time. This means that systems with different framerate will result in different visualisations, but requesting given amount of frames rendered will result in the same lifespan and total number of particles in both systems.

Emitter supports following parameters:

\begin{itemize}
	\item position -- initial position of created particles
	\item angle -- angle counting clockwise from vector [0, 1]
	\item spread -- parameter controlling random differences in initial angle of particles
	\item velocity -- initial velocity of particles, in pixels per frame
	\item velocity spread -- parameter controlling random differences in initial velocity of particles
	\item lifespan -- initial lifespan of particles
	\item productionRate -- amount of particles initialized in each frame
\end{itemize}

A Particle has similar properties:
\begin{itemize}
	\item position
	\item velocity
	\item lifespan
	\item age -- counted in frames, when higher then lifespan particle is removed from system
\end{itemize}

Source code of both implementations is attached in Appendix \ref{cha:sourcecode}.

\section{Implementation with high memory allocation}
\label{sec:particlesinitial}

Initial tested implementation has one very important property of particle emitter. Whenever new particles are created, new array of pointers is allocated and returned from emitter. System appends new particles to existing array. In each frame particle system creates new, empty array of particles and adds there only particles that are still alive. Array from previous frame and all dead particles are removed from system and deallocated. This is clearly suboptimal solution that allocates and deallocates plenty of memory in each frame. Purpose of this exercise is to show how both languages handle bad code and how big impact it has comparing to the optimal solution.

TODO: this part may need to be updated before publication.
TODO: Is this accounting properly for v8 startup time? Maybe profiling ticks would be better.

\lstinputlisting[caption=Time measurement of unoptimized particle system in JavaScript,label=listing:timejs1]{particles/timejs1.txt}
\lstinputlisting[caption=Time measurement of unoptimized particle system in C++,label=listing:timecpp1]{particles/timecpp1.txt}

Time measurement shows that JavaScript version is almost 8 times slower than native one. To analyse situation --prof and --log-timer-events flags may be used. Output file v8.log is parsed using available online tool.\footnote{http://v8.googlecode.com/svn/branches/bleeding\_edge/tools/profviz/profviz.html}

\lstinputlisting[caption=Profiler output for unoptimized particles,label=listing:particles1profile]{particles/particles1-profile.txt}

Methods prefixed with ~ are unoptimized, the ones prefixed with * are JIT compiled. As seen in profiler log, most of the time is spent in unoptimised versions of verifyIfAlive and stepParticle methods of particle system step.

\lstinputlisting[caption=Annotated part of source,label=listing:particles1stepannotated,language=JavaScript]{particles/particleSystemstep.js}

The same methods are also used in optimised versions, where they take significantly less ticks to run. It's clear that presented code not only allocates and deallocates too much memory, but also fails to run in optimised mode. It's visible on chart obtained from the same tool -- stripe labelled "code kind in execution" shows multiple kinds of code running and is interrupted often with garbage collection cycles.

\begin{figure}[h!]
  \caption{Chart of time used in unoptimised verion of JavaScript}
  \label{img:particles1profile}
  \centering
	\includegraphics[width=16cm]{particles/particles1-profile.png}
\end{figure}

Garbage collection cycles blocking execution are also visible with --trace-gc flag.

\lstinputlisting[caption=Garbage collection in unoptimised particle system,label=listing:particles1gc]{particles/particles1gc.txt}

\section{Implementation with object pool}
\label{sec:particlesobjectpool}

To improve performance different approach to particles allocation is used. Each particle has a flag "isDead" telling if it may be safely reused for new particle. Particle pool is kept along with a list of pointers to dead particles. This way when system reaches it's maximum congestion (around 15000 particles in given example) no new allocations occur.
Creation of new particles is moved from particle emitter to particle system, to avoid allocation of new array. Emitter works now as a structure describing behaviour but not implementing one.

\lstinputlisting[caption=Time measurement of optimized particle system in JavaScript,label=listing:timejs2]{particles/timejs2.txt}
\lstinputlisting[caption=Time measurement of optimized particle system in C++,label=listing:timecpp2]{particles/timecpp2.txt}

Optimised version shows overall improvement of 85\% for JavaScript and 45\% for C++ making JavaScript version only 2.2 times slower than native one. It's clearly visible that JavaScript is more sensitive to unwise memory management.

\lstinputlisting[caption=Profiler output for optimized particles,label=listing:particles1profile]{particles/particles2-profile.txt}

Profiling shows that step method of particle system now runs always in optimised mode and almost no time is spent on other methods. The same is visible on profiling chart, where "code kind in execution" stripe shows only optimised code.

\begin{figure}[h!]
  \caption{Chart of time used in optimised verion of JavaScript}
  \label{img:particles2profile}
  \centering
	\includegraphics[width=16cm]{particles/particles2-profile.png}
\end{figure} 

Situation is also improved in garbage collection log.

\lstinputlisting[caption=Garbage collection in optimised particle system,label=listing:particles2gc]{particles/particles2gc.txt}

\chapter{Sphere collision}
\label{cha:spherecollision}

Spheres are the simplest of bounding shapes used in collision detection. This chapter presents tests for two versions of algorithms -- naive $O(N^2)$ approach and with partitioned space. While simpler algorithm has far greater number of collision checks per frame, it allocates almost no memory per frame. More complex method will minimise number of checks, but additional structure and steps added may influence overall execution time in unexpected way.


\begin{figure}[h!]
  \caption{Example rendering of tested sphere collision system}
  \label{img:spheres}
  \centering
	\includegraphics[width=16cm]{spheres/render.png}
\end{figure}

\section{Algorithm description}
\label{sec:spherealgorithmdescription}

Collision detection for spheres is a trivial task. If distance between two spheres is smaller than sum of their radiuses, spheres collide.

$\sqrt{(S_1.x - S_2.x)^2 + (S_1.y - S_2.y)^2 + (S_1.z - S_2.z)^2} < S_1.radius + S_2.radius$

While the equation is simple, with large number N of colliding objects complexity of this detection is $O(N^2)$. Methods of space partitioning are used to reduce number of checks. One used in this benchmark is Octree.
Base for algorithm is a tree-like structure of bounding boxes. Whenever a box contains more than one colliding object, it's divided into eight smaller boxes, by partitioning each edge by 2. When maximum tree depth is reached, multiple objects are stored in one box. One object may be referenced from multiple boxes, when it's size and position make them intersect. Each movement requires a check if object has already moved to one of neighbour boxes.

\begin{figure}[h!]
  \caption{Octree structure. Source: http://en.wikipedia.org/wiki/File:Octree2.svg/}
  \label{img:octree2}
  \centering
	\includegraphics[width=10cm]{octree/octree2.png}
\end{figure} 

Having objects grouped in boxes reduces complexity of collision check. Since an object may collide only with objects in the same box, number of checks is much smaller. Overall complexity of Octree checks is $O(N log{N})$.
TODO: reference for complexity.

\begin{figure}[h!]
  \caption{Example of WebGL Octree debug rendering. Available online at http://pawlowski.it/octtree/}
  \label{img:octree}
  \centering
	\includegraphics[width=10cm]{octree/octree.png}
\end{figure} 

When collision is detected, collision response is calculated. From rule of conservation of momentum:

\begin{center}
$m_1 * \vec{v_1} + m_2 * \vec{v_2} = m_1 * \vec{v'_1} + m_2 * \vec{v'_2}$
\end{center}

Meaning that change of both momentums is of equal value.
\begin{center}
$m_1*\vec{v'_1} =  m_1*\vec{v_1} - \Delta P$

$m_2*\vec{v'_2} =  m_2*\vec{v_2} + \Delta P$

$\vec{v'_1} =  \vec{v_1} - \frac{\Delta P}{m_1}$

$\vec{v'_2} =  \vec{v_2} + \frac{\Delta P}{m_2}$
\end{center}

To simplify response rotation and deformation of spheres are ignored. This does not affect performance analysis, since operations performed all tests is done in the same way.

Let
\begin{center}
$P = |\Delta P|$

$N = \hat{pos_1 - pos_2}$
\end{center}

Since transference of momentum occurs only along single point of contact:
\begin{center}
$ \Delta P = P * \hat{\vec{N}}$

$\vec{v'_1} =  \vec{v_1} - \frac{P}{m_1} * \vec{N}$

$\vec{v'_2} =  \vec{v_2} + \frac{P}{m_2} * \vec{N}$
\end{center}

Lets split each velocity into two scalars, perpendicular and parallel value of velocity vector, and introduce $\vec{Q}$, similar to $\vec{N}$, a perpendicular normalised vector lining along exchanged momentum.

 \begin{center}
$\vec{v_1} =  a_1 * \vec{N} + b_1 * \vec{Q}$

$\vec{v_2} =  a_2 * \vec{N} + b_2 * \vec{Q}$

$\vec{v'_1} =  a'_1 * \vec{N} + b'_1 * \vec{Q}$

$\vec{v'_2} =  a'_2 * \vec{N} + b'_2 * \vec{Q}$
\end{center}

\begin{figure}[h!]
  \caption{Illustration for collision response}
  \label{img:spheresbounce}
  \centering
	\includegraphics[width=8cm]{spheres/bounce.jpg}
\end{figure} 

Deriving from previous equations:

 \begin{center}
$a_1' =  a_1 - \frac{P}{m_1}$

$b_1' = b_1$

$a_2' =  a_2 + \frac{P}{m_2}$

$b_2' = b_2$
\end{center}

Now lets use rule of conservation of energy to solve P:

 \begin{center}
$\frac{m_1}{2} * ||\vec{v_1}||^2 + \frac{m_2}{2} * ||\vec{v_2}||^2 = \frac{m_1}{2} * ||\vec{v'_1}||^2 + \frac{m_2}{2} * ||\vec{v'_2}||^2$

$\frac{m_1}{2} * ({a_1}^2 + {b_1}^2) + \frac{m_2}{2} * ({a_2}^2 + {b_2}^2) = \frac{m_1}{2} * ({a'_1}^2 + {b'_1}^2) + \frac{m_2}{2} * ({a'_2}^2 + {b'_2}^2)2$

$P = \frac{2*m_1*m_2*(a_1-a_2)}{m_1+m_2}$
\end{center}

and finally, using result from conservation of momentum:

 \begin{center}
$\vec{v'_1} =  \vec{v_1} - \frac{2*(a_1-a_2)}{m_1+m_2} * m_2 * \vec{N}$

$\vec{v'_2} =  \vec{v_2} + \frac{2*(a_1-a_2)}{m_1+m_2} * m_1 * \vec{N}$
\end{center}

From this result, using only dot product of velocity vectors and normalised vector $pos_1 - pos_2$ correct response to collision is calculated. In tested scenarios mass of all spheres is equal since it doesn't affect complexity of calculations and produces less randomised results.


\section{{$O(N^2)$} approach}
\label{sec:sphereinitial}

Naive approach for collision detection proves to by easy to implement in JavaScript. Since almost no memory is allocated in each frame, no garbage collection issues appear. All methods are well defined and work mostly on floats. This results in highly optimised binary code produced by compiler, as shown on \ref{img:spheres1profile}.

\begin{figure}[h!]
  \caption{Chart of time used in optimised version of JavaScript}
  \label{img:spheres1profile}
  \centering
	\includegraphics[width=16cm]{spheres/spheres1-profile.png}
\end{figure} 

Multiple tests with N=1000 and different number of frames rendered show, that for simple mathematical task performance of JavaScript is very close to this of C++. On average, JavaScript version of benchmark runs 15\% longer than C++ one.

\begin{figure}[h!]
  \caption{Comparison of total execution time. N = 1000, varying number of frames.}
  \label{img:spheres1-time-total}
  \centering
	\includegraphics[width=16cm]{spheres/time-total.png}
\end{figure} 
\begin{figure}[h!]
  \caption{Comparison of execution time per frame. N = 1000, varying number of frames.}
  \label{img:spheres1-time-per-frame}
  \centering
	\includegraphics[width=16cm]{spheres/time-per-frame.png}
\end{figure}

\section{Octree-partitioned space}
\label{sec:sphereoctree}

\begin{figure}[h!]
  \caption{Octree partitioned sphere collision system}
  \label{img:spheres}
  \centering
	\includegraphics[width=16cm]{spheres/render2.png}
\end{figure}

Tests with Octree partitioning were executed with N=1000 spheres and T=1000 frames. Varying value is maximum depth of Octree, ranging from 1 to 10. 
Changing maximum depth reduces number of collision checks between spheres, as shown on \ref{img:octree-collisions}.

\begin{figure}[h!]
  \caption{Number of collisions in Octree}
  \label{img:octree-collisions}
  \centering
	\includegraphics[width=16cm]{spheres/octree-collisions.png}
\end{figure}

For low values overall complexity of checks doesn't change significantly, since most of spheres are in one or few bounding cubes and no checks are skipped. Additional operations related to Octree are actually making this solution slower than $O(n^2)$ approach. For depth values in optimal zone, number of collision is reduced by factor of at least 10, while keeping Octree overhead reasonable. Interesting thing happens when maximum level of Octree is very high and edge of smallest Octree cube approaches size of spheres. Number of transitions between partitioning cubes, related memory allocation and cleanups actually make this approach much slower, as shown on \ref{img:octree-time}. Moreover, some spheres are references in more than one cube, raising again number of collision checks.

\begin{figure}[h!]
  \caption{Run times in Octree system}
  \label{img:octree-time}
  \centering
	\includegraphics[width=16cm]{spheres/octree-time.png}
\end{figure}

It's clearly visible that number of collision checks and run time is correlated  only up to certain point. For deep Octrees number of checks doesn't improve further, but overall run time is getting longer. Performance of JavaScript in relation to C++ varies between 30\% to 80\% overhead. In comparison with $O(n^2)$ approach, optimal Octree in JavaScript runs over 92\% faster and C++ over 94\% faster. 

\chapter{Emscripten}
\label{cha:emscripten}

JavaScript is often called an assembly language of the Web.\footnote{http://www.hanselman.com/blog/JavaScriptIsWebAssemblyLanguageAndThatsOK.aspx} One could argue that since only one language is supported by browsers it could be made a compilation target similar to assembler for CPU. This statement is flawed since eventually JavaScript is translated to assembly making it only an intermediate step. Probably resemblance to ByteCode in JVM, which is compilation target of multiple languages like Java, Scala and Clojure is more in place.
Nevertheless, last years showed multiple projects aimed at converting code to JavaScript. Some introduce new syntax like CoffeeScript, Dart or TypeScript while still serving the same purpose - providing human readable code that is interpreted in browser on fly. Others, that are focus of this chapter, aim to convert existing projects to run in browser.

Several new projects are connected to make this happen. First steps in conversion between languages were made with LLVM project\footnote{http://llvm.org/} which currently is a collection of tools and compilers converting code to and from intermediate representation (LLVM IR). For C++ Clang\footnote{http://clang.llvm.org/} is a conversion tool.

\begin{figure}[h!]
  \caption{Pipeline of Emscripten conversion. Source: http://www.hanselman.com/blog/JavaScriptIsWebAssemblyLanguageAndThatsOK.aspx}
  \label{img:emscriptenpipeline}
  \centering
	\includegraphics[width=8cm]{emscripten/pipeline.jpg}
\end{figure}

Code in LLVM is suitable for further conversion to language like JavaScript. This part is handled by Emscripten\footnote{https://github.com/kripken/emscripten/wiki} project. Initially compilation target for Emscripten was plain JavaScript. With recent developments asm.js\footnote{http://asmjs.org/spec/latest/} library was created. It provides syntax built on top of JavaScript, that is strongly typed and easily translatable to assembly language. All language specific annotations are added through bit NOP operations.

\lstinputlisting[caption=Example of code using asm.js,label=listing:asmjs]{emscripten/asm.js}

Project, built in cooperation with Mozilla Foundation, has its own engine for Firefox - OdinMonkey, designed to run faster for this limited and well-defined syntax.

Altogheter these projects resulted in multiple libraries and games converted from native version to JavaScript.

Proof-of-concept demo made in cooperation between Mozilla and Unreal is Epic Citadel HTML5 - Unreal Engine 3 technology demo\footnote{http://www.unrealengine.com/en/showcase/udk/epic\_citadel/}  instance running in browser.\footnote{http://www.unrealengine.com/html5/} Companies claim it took only four days to complete the conversion.

\begin{figure}[h!]
  \caption{Epic Citadel screenshot}
  \label{img:epicitadel}
  \centering
	\includegraphics[width=16cm]{emscripten/epic-citadel.jpg}
\end{figure}

Another example of successful converted project is ammo.js\footnote{https://github.com/kripken/ammo.js/} - originating from Bullet physics engine.
TODO: Maybe extend this part a bit, cover more on how conversion was going and what were the issues.

\begin{figure}[h!]
  \caption{Ammo.js demo colliding 500 boxes at 30fps, available at http://kripken.github.io/ammo.js/examples/new/ammo.html}
  \label{img:epicitadel}
  \centering
	\includegraphics[width=16cm]{emscripten/ammojs.png}
\end{figure}

\section{Technology overview}
\label{sec:emscriptenoverview}

TODO: Explain static memory allocation in Float32Array and consequences for memory management.
TODO: Overview of asm.js modules, ahead of time compilation.


\section{Benchmarks}
\label{sec:embenchmarks}

TODO: Gather more benchmark results after codebase is frozen.
TODO: Summary of results - memory intensive demos work faster in Emscripten/asm.js, the ones with close to zero GC are faster in JS. Consider how usage of typed arrays would change results.

\begin{table}[h!]
\caption{Particle tests on different platforms}
\label{table:benchmarks}
\begin{tabular}{|p{4cm}||l|l|l||l|l|l|}
  	\hline
   Platform & \multicolumn{3}{c}{Unoptimised particles} & \multicolumn{3}{c}{Optimised particles}\\ \hline
   & C++ & JavaScript & Emscripten & C++ & JavaScript & Emscripten\\ \hline
   Fedora 19, Intel i7 2670QM, 4GB RAM, g++ 4.8.1 & 3.21s & 19.51s & 4.85s & 1.63s & 4.96s & 5.10s \\ \hline
   Windows 7, Intel i7 2670QM, 4GB RAM, g++ 4.7.3, Cygwin & 3.21s & 20.45s & 6.18s & 1.47s & 3.28s & 5.39s \\ \hline
\end{tabular}
\end{table}

\begin{table}[h!]
\caption{Spheres tests on different platforms}
\label{table:benchmarks}
\begin{tabular}{|p{4cm}||l|l|l||l|l|l|}
   \hline
   Platform & \multicolumn{3}{c}{$O(n^2)$ spheres} & \multicolumn{3}{c}{Octree spheres}\\ \hline
   & C++ & JavaScript & Emscripten & C++ & JavaScript & Emscripten\\ \hline
   Fedora 19, Intel i7 2670QM, 4GB RAM, g++ 4.8.1 & 4.96s & 9.02s & 12.35s & 
3.44s & 14.14s & 11.20s \\ \hline
   Windows 7, Intel i7 2670QM, 4GB RAM, g++ 4.7.3, Cygwin & 9.26s & 10.79s & 11.19s & 13.87s & 13.19s & 9.71s \\\hline
\end{tabular}
\end{table}

\chapter{Summary}
\label{cha:summary}

Conducted experiments show a gap between JavaScript and C++ performance. Significant language design differences result in code that is often easier to write but also easier to abuse. Benchmarks show differences between 15\% and 100\% overhead for correctly designed JavaScript code and over 500\% for incorrect patterns.  Considering Moore's law stating that computers double speed every 18 months it safe to say that JavaScript is very close to being suitable for any type of development. Recent projects varying from server side solutions\footnote{http://nodejs.org/}\footnote{http://googlecreativelab.github.io/coder/} to hardware developer boards\footnote{http://www.espruino.com/} are proving it. From the perspective of game development, it is unlikely at the time of writing that AAA game may be created to run in browser. However, growing segments of casual, independent and social games are already targeting web as a platform.

\begin{figure}[h!]
  \caption{Game created with ImpactJS}
  \label{img:impactjs}
  \centering
	\includegraphics[width=12cm]{summary/impactjs.png}
\end{figure}

Multiple open-source and commercial game engines are created lately. Examples worth mentioning are ImpactJS\footnote{http://impactjs.com/}, Turbulenz\footnote{http://biz.turbulenz.com/turbulenz} and Isogenic Engine\footnote{http://www.isogenicengine.com/}.


\begin{figure}[h!]
  \caption{Game created with Isogenic Engine}
  \label{img:isogenic}
  \centering
	\includegraphics[width=12cm]{summary/isogenic.png}
\end{figure}

Very important and growing sector are interactive 3D arts with two major targets - music videos and commercials. They are uniquely available only in browsers as a very viral extensions of normal marketing. On of first occurrences of this use of technology was video for Rome group: "3 dreams of black"\footnote{http://www.ro.me/}. Project allows to move the camera while animated 3D story is rendered for music. After movie is over user is allowed to create 3D models that are later incorporated into experiences of next people watching. This way interactions and social element are enabled in what used to be one-way transmission of art form. 

\begin{figure}[h!]
  \caption{Screenshot from "3 dreams of black"}
  \label{img:rome}
  \centering
	\includegraphics[width=12cm]{summary/rome.jpg}
\end{figure}

\section{Encountered environment limitations}
\label{sec:limitations}

\section{Browser advantages}
\label{sec:advantages}

\section{Recommended techniques}
\label{sec:recommended}

\section{Final thoughts}
\label{sec:final} 

\appendix
\chapter{Acknowledgements}
\label{cha:acknowledgements}



\chapter{Source code}
\label{cha:sourcecode}

\section{Particle system}
\label{sec:particlesource}

\lstinputlisting[caption=Particle object in JavaScript,label=listing:particlejs,language=JavaScript]{particles/particle.js}
\lstinputlisting[caption=Particle object in C++,label=listing:particlejs,language=C++]{particles/particle.cpp}
\lstinputlisting[caption=Particle emitter object in JavaScript,label=listing:particleemitterjs,language=JavaScript]{particles/particleEmitter.js}
\lstinputlisting[caption=Particle emitter object in C++,label=listing:particleemittercpp,language=C++]{particles/particleEmitter.cpp}
\lstinputlisting[caption=Initial particle system object in JavaScript,label=listing:particlesystem1js,language=JavaScript]{particles/particleSystem.js}
\lstinputlisting[caption=Initial particle system in C++,label=listing:particlesystem1cp,language=C++]{particles/particleSystem.cpp}
\lstinputlisting[caption=Optimised particle system object in JavaScript,label=listing:particlesystem2js,language=JavaScript]{particles/particleSystem2.js}
\lstinputlisting[caption=Optimised particle system in C++,label=listing:particlesystem2cpp,language=C++]{particles/particleSystem2.cpp}


\clearpage
\addcontentsline{toc}{chapter}{References}
\bibliography{bibliografia}

\end{document}
 
